\documentclass[11pt,a4paper]{article}
    \usepackage{jinstpub}
    \usepackage{floatrow}
    \usepackage{subfig}
    \title{Quality control and batch testing for STAR-EPD front-end readout module}
    \author[a]{Liang Z.}
    \author[a]{Shen K.}
    \author[a]{Wu Y.}
    \author[a]{Shao M.}
    \affiliation[a]{University of Science and Technology of China, JinZhai Road, HEFEI, China}

    \emailAdd{liangzheng1021@163.com}
    \emailAdd{skfyl@mail.ustc.edu.cn}
    \emailAdd{torrence@mail.ustc.edu.cn}
    \emailAdd{swing@ustc.edu.cn}

    \abstract{%\bf R\mdseries elativistic \bf H\mdseries eavy \bf I\mdseries on \bf C\mdseries ollider(RHIC), which locates at \bf B\mdseries rookhaven 
    % \bf N\mdseries ational \bf L\mdseries ab(BNL), is determined to collide ions. \bf S\mdseries olenoidal \bf T\mdseries racker \bf A\mdseries t \bf R\mdseries HIC(STAR)
    % tracks the thousands of particles produced by ion collisions at RHIC. 
    \emph{E}vent \emph{P}lane, centrality, and trigger \emph{D}etector(EPD) is a plastic scintillation detector
    designed to be installed on RHIC-STAR. SiPMs are used to collect photons produced by scintillators, while FEE boards and RX boards are used to amplify and recieve signals from SiPMs.
    We care about their performances before installing, e.g., UI curve, noise frequency spectrum, signal characters.
    Thus, we setup a batch test system, which is made up of pulse generator, LED, digitizer, etc.
    }

    \keywords{STAR, EPD, SiPM, FEE board, RX board, Noise, Signal, Batch test}
    
    % \notoc
    % \toccontinuoustrue

\begin{document}
\maketitle
\flushbottom

\section{Introduction}
Relativistic Heavy Ion Collider (RHIC), locates at Brookhaven National Lab (BNL), is determined to collide heavy ions, which led to the discovery of Quark-Gluon Plasma (QGP).
The Solenoidal Tracker at RHIC (STAR) tracks particles produced by RHIC and is able to make important measurements helping answer key questions for understanding QCD matter and strong force.
The Beam Energy Scan program (BES) at RHIC has shown hints of a critical point and first-order phase transition at the BES energies.
Key measurements for locating the critical point and determining the first order phase transition are limited by poor event plane resolution, limited statistics, and a TPC-only centrality determination.
Therefore, phase II of the BES program was proposed to take data with upgraded detectors and increased statistics for the further investigation.\cite{pr17}\cite{cyang}

A new event plane and collision centrality detector is planned to replace the existing detector, the Beam-Beam Counter (BBC), with higher granularity and acceptance.
The design of the Event Plane Detector (EPD) consists of two 1.2-cm-thick scintillator discs at $z= \pm 3.75~m$ from the center of STAR, covering $2.2 <\eta< 5.1$, the same as the BBC.
To maximize event plane resolution, centrality estimation and flow harmonic measurements, the disc is separated into 372 tiles (12*2 azimuthal sectors, 16+15 radial segments, see [Fig. 1]).
Each tile is embedded with 3 turns of wavelength shifting fibers coupled to clear fiber outside for readout.
The emitted photons will be read out by silicon photomultipliers (SiPM) - an inexpensive and magnetic field insensitive replacement for the traditional photo-multiplication tube, which also enhance STAR triggering capability with a nanosecond level timing resolution.

A prototype consisting of a single sector was integrated into STAR during the 2016 run and 1/8 super-sectors in 2017 run. The full EPD construction proposal was approved in 2017, while the final design was confirmed. [Refer.]
To catch up with the commissioning run in 2018 , USTC had undertook the challenging tasks for mass production and quality control of front-end readout module. which is comprised of SiPM board, front-end electronics (FEE) board, and receiver(RX) board.
Results from this first process during the construction of full EPD would be a baseline of detector performance.
In this paper, characterization of every components was researched and parameters were selected for criteria. For better efficiency and specialized operation , we developed a dedicated system for the quality control and batch test, including test platform, GUI controller, and web database. With optimized workflow and control software, we were able to achieve the extreme uniformity across all 744 channels. [JEwig-QM2018]

\section{Front-end Readout Module}
To integrated with the STAR mechanic structure and Data Acquisition system (DAQ) [Refer], the front-end readout module of EPD is well-modularized for photo-electric conversion (SiPM), power and temperature management (FEE) as well as signal signal transmission (RX).
Considering the space of installation is rarely small, the RX board was designed as an adapter to interface between the triggering system and the FEE box ($\sim 0.5~m^3$).
For compactness, each module processes 16 channels of fiber signal, which is  attached to the SiPM board with a fiber-to-SiPM connector. [Fig. 2]
\subsection{SiPM and multi-channel board}
It's a board aiming at sticking 16 SiPMs together, supplying bias voltage and reading out signals. It has a row of gold-fingers, which makes the board easily inserted into FEE board. Picture of a board with 16 SiPMs stuck to is as shown in Figure 1.
In order to ensure the PCB boards' quality, appearance of which should be
examined under microscope. Because SiPMs are too small to stick by ourselves, thus we have it done by a factory, which has professional machines. However, they may also make mistakes. It's our duty to pick them out. First, all SiPMs should not have any dust or scratch and be installed properly. All pins of a SiPM should be sticked to PCB board firmly on right place. Second, we need to make sure that distance between upper surface of SiPM and upper surface of PCB board is within the proper range. Too much variation may bring difficulties to SiPMs' coupling with fibers. There's a spacer, as is shown in 1, between end fiber connector and SiPM Board, which is printed by 3-D printer and have a adjustable thickness, and is used to protect SiPMs from mechanic damages when coupling with fibers through air. However, if the height of SiPM varies a lot, the distance between fiber and SiPM is different. This makes it hard to maintain light uniformity. These can be visual inspected under microscope.
\subsection{FEE board}
Front-End Electronic(FEE) board is designed to communicate with computer,
monitor and control bias voltage supply, amplify and output signals from SiPMs.
There’re several connectors that have different functions. One is for power supply, including low voltage for electronics in FEE Board and bias voltage for SiPM board. One is prepared for SiPM board, which should be inserted into FEE Board. One is for outputting amplified signals. Amplified signals will be sent to another PCB board for further process's convenience. Another one is for communicating with computer. Bias voltage supplied to SiPM can be monitored and controlled through computer, meanwhile, current of SiPM can also be monitored. Besides, output signals’ offsets can also be adjusted through computer. Temperature compensating for bias voltage is also considered in FEE Board.
FEE Board is the most sophisticated one among the three kinds of board. Meanwhile, because its complexity, we don't have enough way to judge its functioning except by inferring from final output signal from RX board. This will be mentioned latter.
\subsection{RX board}
Due to high integration of FEE Board, there's no sufficient space for standard
connectors. Thus, Receiver(RX) Board is designed to receive signals from FEE Board and output through SMA standard connectors. RX Board is connected with FEE Board through a IEEE-1284 Compliant Parallel Cable, on which signals from SiPMs are transmitted. Then those signals will be sent to SMB connector. This will make it possible to show these signals in oscilloscope or other instruments. Detailed examination of RX board is beyond our ability. However, we can infer its
performance from its output signals. Besides, final output signals are determined by both FEE Board and RX Board, therefore, analysis of final signals can also help us judge functioning of FEE Board.
First, we concern about their noise. Lower bias voltage down to breakdown voltage of SiPMs, which is 46.1~V in our case. SiPM won't output any signal, thus, we can analyze noise of electronics. We can get noise frequency spectrum from oscilloscope in real time. Or, relatively, record several waveforms and analyze latter. Anyway, we need to ensure that the spectrum is smooth and don't have unreasonable rips.
Second, we need to measure offsets of output signals. FEE Board is able to adjust offsets in a certain range. Thus, we need to measure several offsets, under different given control parameters in computer, to get offset range and relation to control parameter.
Most importantly, we need to analyze signals from SiPMs. Although signals from SiPMs have lots of characteristics, we only concern about single photon signal. We need to see single photon peak in charge spectrum.

\section{System Design}
\paragraph{Test Items \& Test Methods}First, all device's mechanical integrity should be ensured and SiPMs' thickness should be measured,
i.e., we should first check whether there exists any obvious damage and than measure thickness of SiPM by microscope.

Second, UI curve of a SiPM can help us judge its basic electric properties, e.g., dark current and break down(BD) voltage.
Under good dark conditions, we change bias voltage on SiPMs from 50V to 65V, with 0.5V step size. This range covers all what we are interested in.
We can control FEE board and acquire U/I information through computer.

Third, electronic noise can reflect electronic performances. Rib or strange peaks shouldn't appear in noise frequency spectrum. Thus, we set bias voltage on
SiPM at 46.5V, which is below its BD voltage, and output RX board signal into oscilloscope or digitizer to collect waveforms, and finally get frequency spectrum.

Last, several characters of SiPM's signal should be tested. Shape of signal should be normal, and pedestal should be changed when adjusting DAC. In addition, we need check
spectrum of signal charge integral in order to judge the quality of single photon signal's resolution. We first set bias voltage below BD voltage(46.5V) to check pedestal. And
then, set bias voltage at OP voltage(60V) to check integral spectrum, meanwhile, illuminate SiPMs with light from a LED, which is driven by wavefrom generator. Output all signal
into oscilloscope/digitizer, which is triggered in the same frequency as LED. After collecting enough waveforms and a little further processing, we get integral spectrum.

\paragraph{System Chart}
System chart is as shown in figure \ref{fig:System Chart}.

\paragraph{Instrument Selection}It's easy to choose LED, whose luminous wavelength range should cover peak of scintillators' luminescence spectrum, i.e., 475nm.
However, choosing oscilloscope or digitizer is a tough decision. Oscilloscope can help to get more accurate result, but only have 4 input channels. While digitizer have
16 channels but has less precision. Besides, digitizer should be calibrated before using.

\begin{figure}[ht]
    \centering
    \begin{floatrow}
        \centering
        \ffigbox{\caption{System Chart}\label{fig:System Chart}}{\includegraphics[scale=0.45]{fig/System_Chart.pdf}}
        \centering
        \ffigbox{\caption{UI Curve Example}\label{fig:UI Curve Example}}{\includegraphics[scale=0.4]{fig/UI_Curve.pdf}}

    \end{floatrow}
    % \includegraphics[scale=0.25]{Flow_Chart.png}
\end{figure}



\section{SiPM Characterization}
    \paragraph{UI Curve}
    %we measure UI characteristic curve of each SiPM through current and voltage monitors, which are installed on FEE Boards. 
    
    % \begin{figure}[!h]
    %     \centering
    %     \graphicspath{{/Users/john/Desktop/EPD_Summary/UI_Curve/}}
    %     \includegraphics[scale=0.2]{Characteristic_UI_Curve_Example.jpg}
    %     \caption{UI Curve Example}\label{fig:UI Curve Example}
    % \end{figure}
    Typical UI curve of SiPM is like figure \ref{fig:UI Curve Example}, which shows UI curves of all 16 SiPMs in one FEE board.
    
    Due to low accuracy of current monitor, current measurement is less precise than 0.01$\mu$A.
    Thus, we can only get dark current quite roughly. Anyway, we can see a sudden increase when bias voltage increase.
    We choose the point, whose current is closest to 0.1 $\mu$A as break point, for the purpose of simplifing.
    At last, we can get distribution of "break down voltage".
 

    \paragraph{Noise} Comparison between oscilloscope and digitizer is showed in figure \ref{fig:Noise Comparison}
    % \begin{figure}[!h]
    %     \graphicspath{{/Users/john/Desktop/EPD_Summary/Noise/}}
    %     \ffigbox[\FBwidth]{}{
    %         {
    %             \begin{subfloatrow}[2]
    %                 \ffigbox[\hsize]{\caption{Oscilloscope noise}}{\includegraphics[scale=0.15]{Characteristic_Osc_Noise.jpg}}
    %                 \ffigbox[\hsize]{\caption{Electronic noise analysed by oscilloscope}}{\includegraphics[scale=0.15]{Characteristic_Noise_Osc.jpg}}
    %             \end{subfloatrow}    
    %         }
    %         \caption{Results from oscilloscope}\label{hello}
    %     }
    %     \ffigbox[\FBwidth]{}{{
    %     \begin{subfloatrow}[2]
    %         \ffigbox[\hsize]{\caption{Digitizer noise}}{\includegraphics[scale=0.15]{Characteristic_Dig_Noise.jpg}}
    %         \ffigbox[\hsize]{\caption{Electronic noise analysed by oscilloscope}}{\includegraphics[scale=0.15]{Characteristic_Noise_Dig.jpg}}
    %     \end{subfloatrow}}\caption{Results from digitizer}}


    %     % \end{floatrow}
    % \end{figure}

    \begin{figure}[ht]
        \centering
        \subfloat[Electronic Noise by Oscilloscope]{\includegraphics[scale=0.3]{fig/Noise_Osc.pdf}}\hspace{20pt}
        \subfloat[Electronic Noise by Digitizer]{\includegraphics[scale=0.3]{fig/Noise_Dig.pdf}}\
        \subfloat[Oscilloscope Noise]{\includegraphics[scale=0.15]{fig/Osc_Noise.png}}
        % \subfloat[Digitizer Noise]{\includegraphics[scale=0.15]{_Dig_Noise.jpg}}\hspace{20pt}
        \caption{Comparison between Dig \& Osc}\label{fig:Noise Comparison}
    \end{figure}

    It seems that results from oscilloscope are more accuracy, while results from digitizer only have resolution of 1 MHz.
    Strangely, there still exist several ribs in oscilloscope noise spectrum when no signal input at all.
    We believe it's blame to oscilloscope itself.
    Considering of tight schedule, we deside to use digitizer to setup batch test system.
        
        % \centering
        % \includegraphics[scale=0.2]{Characteristic_Noise_Osc.jpg}
        % \caption{Noise spectrum measured by oscilloscope}

        % \includegraphics[scale=0.2]{Characteristic_Noise_Dig.jpg}
        % \caption{Noise spectrum measured by digtizer}
    % \end{figure}


    \paragraph{Signal}
    \begin{figure}
        \centering
        % \subfloat[Signal from Oscilloscope]{
            \ffigbox[\FBwidth]{}
            {
                {
                    \begin{subfloatrow}[2]
                        \ffigbox[\FBwidth]{\caption{Signal from Oscilloscope}}{\includegraphics[scale=0.45]{fig/Signal_Osc2.pdf}}
                        % \ffigbox[\FBwidth]{\caption{Signal Persistence}}{\includegraphics[scale=0.2]{Signal_Osc.pdf}}\
                        \ffigbox[\FBwidth]{\caption{Signals from Digitizer}\label{fig:Digitizer signal example}}{\includegraphics[scale=0.35]{fig/Signal_Dig.pdf}}
                    \end{subfloatrow}
                }
                \caption{Signal Examples}\label{fig:Signal Example}

            }
        % }
        % \subfloat[Signal from Digitizer]{\includegraphics[scale=0.2]{Signal_Dig.pdf}}

        % \caption{Signal Examples}\label{fig:Signal Example}

        % \includegraphics[scale=0.2]{Signal_Osc.pdf}
        % \caption{Signal Example}\label{fig:Signal Example}
    \end{figure}
    Signal examples are as shown in figure \ref{fig:Signal Example}.

    \subparagraph{Pedestal}As is shown in figure \ref{fig:Digitizer signal example}, pedestals of signals from digitizer are sloping.
    With further disicussion, we find that different pedestals have different slope.
    However, differece of slopes are too small to influnce pedestal value.
    Thus, it's a reasonable convention that average value of pedestal is its measurement result.

    \begin{figure}[ht]
        % \ffigbox[\FBwidth]{}{
        %     {
        %         \begin{subfloatrow}[2]
        %             \ffigbox{\caption{Result from oscilloscope}}{\includegraphics[scale=0.2]{Oscilloscope_Pedestal.jpg}}
        %             \ffigbox{\caption{Result from digitizer}}{\includegraphics[scale=0.2]{Digitizer_Pedestal.jpg}}
        %         \end{subfloatrow}
        %     }

        \includegraphics[scale=0.5]{fig/Pedestal.pdf}
        \caption{Pedestal Testing}\label{fig:Pedestal Testing}
        % }
    \end{figure}


    % Figure \ref{fig:Pedestal Testing} shows relation between DAC and pedestal value. We can see that DAC can work well.
    From figure \ref{fig:Pedestal Testing} we can see adjusting range of pedestal.





    \subparagraph{Charge Intergral of Signal}

    We collect more than 5000 signals from oscilloscope/digitizer and then integrate time and amplitude. Finally, we get its
    spectrum like figure \ref{fig:Integral Testing}.

    There's a little trick when analysing waveforms from digitizer. Assuming the signal interval that contains whole SiPM signals activated
    by photons from LED is 550ns to 650ns.
    We choose 430ns and 530ns as our first reference interval, and 670ns to 770ns as second reference interval. 
    Average of this two reference intervals can be regard as pedestal value of signal interval's midpoint.
    Minusing the average value when integrating can totally get rid of influnce of pedestal slope, and increase resolution of photon spectrum.

    Furthermore, we need to abandon waveforms which contain signal in reference intervals. It's easy to do that if we compare the slope, which is 
    calculated by two average value of two intervals, with the distribution of slope. If the calculated slope is far away from mean value of 
    pedestal distribution, this waveform should be dump.

    \begin{figure}[ht]
        \ffigbox[\FBwidth]{}{
            {
                \begin{subfloatrow}[2]
                    \ffigbox{\caption{Result from oscilloscope}}{\includegraphics[scale=0.4]{fig/Oscilloscope_Integral.pdf}}
                    \ffigbox{\caption{Result from Digitizer}}{\includegraphics[scale=0.4]{fig/Digitizer_Integral.pdf}}
                \end{subfloatrow}
            }
            \caption{Integral Testing}\label{fig:Integral Testing}
        }
        % \includegraphics[scale=0.2]{Integral_Example.jpg}
        % \caption{fjldksajf}
    \end{figure}

    
    % \begin{figure}[ht]
    %     \graphicspath{{/Users/john/Desktop/EPD_Summary/Signal/Integral/}}
    %     \centering
    %     \includegraphics[scale=0.2]{Dig_Signal_Process.jpg}
    %     \caption{Process signal from digitizer}
    % \end{figure}

    



\section{Batch Test}
% \paragraph{Digitizer Calibration}Due to tight time schedule, we decide to use digitizer instead of oscilloscope.
% However, calibration of digitizer should be done before we setup batch test system.
% Have pulse generator generated pulses with different amplitudes. And output them seperately to digitizer and oscilloscope.
\section{Summary}
% \subsection{}


\end{document}
